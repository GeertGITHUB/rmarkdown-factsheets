% set font encoding for PDFLaTeX or XeLaTeX
\usepackage[utf8]{inputenc}
\usepackage[default]{lato}
\usepackage[T1]{fontenc}
\usepackage{color} 
\usepackage{mdframed}
\usepackage{multicol}
\usepackage{graphicx}
\usepackage{booktabs}
\usepackage{longtable}
\usepackage{array}
\usepackage{multirow}
\usepackage[table]{xcolor}
\usepackage{wrapfig}
\usepackage{float}
\usepackage{colortbl}
\usepackage{pdflscape}
\usepackage{tabu}
\usepackage{threeparttable}
\usepackage[normalem]{ulem}
\usepackage{tikz}
\usetikzlibrary{calc}
\usepackage{lipsum}
\definecolor{urbanblue}{HTML}{1696D2}
\usepackage[many]{tcolorbox}

\usepackage{enumitem} % bullet alignment
\setlist[2]{nosep} % sets the itemsep and parsep for all level two lists to 0
\setenumerate{nosep} % sets no itemsep for enumerate lists only


% turn off page numbering
\pagenumbering{gobble}


% contact information
\newcommand{\contactinfo}[0]{
  \thispagestyle{empty}
  \begin{tikzpicture}[remember picture, overlay]
    \draw 
      node[anchor=south west, align=center, minimum height=0.5in, minimum width=8.5in, text width=8in] at ($(current page.south west)$) 
      {\color{gray}Urban Institute
      \color{urbanblue}$\blacksquare$
      \color{gray}2100 M Street NW 
      \color{urbanblue}$\blacksquare$ 
      \color{gray}Washington, DC 20037 
      \color{urbanblue}$\blacksquare$ 
      \color{gray}202.833.7200 
      \color{urbanblue}$\blacksquare$ 
      \textcolor{urbanblue}{www.urban.org}};
  \end{tikzpicture}
}

\usepackage{memhfixc}

% titles (14pt font)
\newcommand{\urbantitle}[1]{
  \begin{center}
      \textbf{\LARGE{#1}}\vspace{-3ex}
  \end{center}
}
  
% subtitles (12pt font)
\newcommand{\urbansubtitle}[1]{
  \begin{center}
    \Large{\textcolor{urbanblue}{#1}}\vspace{-1.5ex}
  \end{center}
}

% authors (11pt font)
\newcommand{\urbanauthors}[1]{
  \begin{center}
    \textit{\large{#1}}
  \end{center}
}


% bullet points
\newenvironment{urbanbullets}
{ \begin{itemize}[align=left]
\setlength\itemsep{0em}
    \setlength{\itemsep}{0pt}
    \setlength{\parskip}{0pt}
    \setlength{\parsep}{0pt}     }
{ \end{itemize}                  } 

% numbered list
\newenvironment{urbanenumerate}
{ \begin{enumerate}[align=left]
\setlength\itemsep{0em}
    \setlength{\itemsep}{0pt}
    \setlength{\parskip}{0pt}
    \setlength{\parsep}{0pt}     }
{ \end{enumerate}                  } 

% funderstatement
\usepackage[none]{hyphenat}

%\newcommand{\urbanboilerplate}[3]{
%  \thispagestyle{empty}
%  \begin{tikzpicture}[remember picture, overlay]
%    \draw (2, 2) to (8, 2);
%    \draw 
%      node[anchor=south west, align=left, minimum height=1.5in, minimum width=8.5in, text width=6.5in] at ($(current page.south west)$) 
%      {This fact sheet was funded by #1. The views expressed are those of the author/authors and should not be attributed to the Urban Institute, its trustees, or %its funders. Further information on the Urban Institute’s funding principles is available at urban.org/fundingprinciples. 
%      
%      Copyright © #2 #3. Urban Institute. Permission is granted for reproduction of this file, with attribution to the Urban Institute.};
%  \end{tikzpicture}
%}

\newcommand{\urbanboilerplate}[3]{
  \vspace*{\fill}
  \noindent\rule{6.5in}{1pt}
  This fact sheet was funded by #1. The views expressed are those of the author/authors and should not be attributed to the Urban Institute, its trustees, or its funders. Further information on the Urban Institute’s funding principles is available at \href{https://www.urban.org/aboutus/our-funding/funding-principles}{urban.org/fundingprinciples}. 

  Copyright © #2 #3. Urban Institute. Permission is granted for reproduction of this file, with attribution to the Urban Institute.
}
